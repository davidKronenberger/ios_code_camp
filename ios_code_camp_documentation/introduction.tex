\section{Einleitung}
Dies ist die Dokumentation zur Applikation \glqq Nice to Have\grqq{ }welche im Zuge des Moduls \glqq (iOS) Code-Camp Context-Awareness 1\grqq{ }im Wintersemester 2016, am Lehrstuhl für Kommunikationstechnik (Chair for Communication Technology – ComTec) angefertigt wurde. Dieses Dokument enthält neben der Aufgabenstellung eine Übersicht und Erklärung zur Implementierung und Umsetzung der Applikation, sowie einen Ausblick über offene Punkte und mögliche weitere Arbeitspakete für eine Weiterentwicklung der Applikation in der Zukunft.

\subsection{Aufgabenstellung}
Ziel des iOS-Codecamps ist, der eigenständige Erwerb von Programmierfähigkeiten im Bereich des von Apple entwickelten Betriebssystems iOS. Innerhalb der Blockveranstaltung sollen die Teilnehmer sich Kenntnisse in der objektorientierten Programmierung, speziell mit der Sprache Objective-C, aneignen und in Gruppengrößen von bis zu fünf Studenten eine eigene Applikation programmieren. 
Bei der Anwendung soll es sich um einen sogenannten \glqq Instant Messenger\grqq{ }handeln, wie man ihn bereits von Applikationen wie \glqq Whatsapp\grqq{ }oder \glqq Signal\grqq{ }her kennt.
Die einzig fest vorgegebene Anforderung war lediglich, dass Textnachrichten versendet und empfangen werden können. Sämtliche weitere Funktionalitäten konnten als optional eingebunden werden. Darüber hinaus stand es den Teilnehmern offen ein komplett eigenes Backend für die Anwendung zu erstellen oder eine bereits fertige Lösung zu verwenden.\newline
\newline
Als Leistungsnachweis gilt es neben der Abgabe des Quellcodes innerhalb einer Präsentation die entwickelte Applikation vorzustellen und eine entsprechende Dokumentation anzufertigen. 
\newpage 