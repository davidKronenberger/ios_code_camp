\section{Ausblick}
Zusammenfassend kann gesagt werden, dass die entstandene Messenger-Applikation alle Anforderungen erfüllen könnte, welche sich die Gruppe als Ziel gesetzt hatte. Besonders der Umgang mit Firebase hatte die dazugehörige Entwicklungszeit durch die bereitgestellten Features, wie Realtime Database oder einen Service zur Authentifikation, stark verringert. 
Es darf dabei nicht außer Acht gelassen werden, dass diese Applikation nur eine Handvoll von den Features eingesetzt, welche Firebase wirklich zur Verfügung hat. Beispielsweise können leicht Funktionalität wie das Löschen oder Editieren einer Gruppe hinzugefügt werden, ohne dabei aufwändige Änderungen am vorhandenen Quellcode durchführen zu müssen. 
Der enstandende Messenger wurde der Fokus stärker auf die Stabilität und Zuverlässigkeit von Funktionen gelegt, als nur neue ungetestete Funktionen hinzuzufügen, um als Resultat eine abgerundete saubere Applikation zu erhalten. Aufbauend zu diesem Messenger können dadurch weitere gewünschte Funktionalitäten leichter ergänzt werden. Dies kann zum Beispiel das Versenden von Sprachnachrichten sein, oder das Setzen von eigenen Gruppenbildern.
